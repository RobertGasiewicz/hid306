\documentclass[sigconf]{acmart}

\usepackage{hyperref}

\usepackage{endfloat}
\renewcommand{\efloatseparator}{\mbox{}} % no new page between figures

\usepackage{booktabs} % For formal tables

\settopmatter{printacmref=false} % Removes citation information below abstract
\renewcommand\footnotetextcopyrightpermission[1]{} % removes footnote with conference information in first column
\pagestyle{plain} % removes running headers

\begin{document}
\title{Internet of Things (IoT) alliance with Big Data}


\author{Murali Cheruvu}
\orcid{xxxx-xxxx-xxxx}
\affiliation{%
 \institution{Indiana University}
 \streetaddress{3209 E 10th St}
 \city{Bloomington} 
 \state{Indiana} 
 \postcode{47408}
}
\email{mcheruvu@iu.edu}

% The default list of authors is too long for headers}
\renewcommand{\shortauthors}{M. Cheruvu}


\begin{abstract}
This paper provides an introduction to Internet of Things (IoT) and how Big Data can effectively improve IoT process.
\end{abstract}

\keywords{Internet of Things, IoT, Big Data, Data Science}


\maketitle

\section{Introduction}

The Internet of things (\textbf{IoT}) is the network of physical devices, vehicles, and other items embedded with electronics, software, sensors, actuators, and network connectivity which enable these objects to collect and exchange data. Devices of all types - cars, manufacturing equipment, medical devices and more - have become smarter, opening up the need for their connectivity with the internet. Today, over 50\% of IoT activity is centered in manufacturing, transportation, smart city, consumer applications like home automation and wearable gadgets, but within five years all industries will have rolled out IoT initiatives. \textbf{Gartner, Inc}. forecasts that 8.4 billion connected things will be in use worldwide in 2017, up 31 percent from 2016, and will reach 20.4 billion by 2020. 

\section{IoT Intuition}

The rise of IoT changes everything by enabling \textbf{smart} things: smart products - cars, airplanes, trains and smart environments - cities, hospitals, schools and stores. Broadly speaking, two kinds of IoT are emerging: Consumer IoT and Industrial IoT. Products such as Apple Watch, Fitbit and Home Automation - TV, thermostats, alarm system, etc. are some of the examples of Consumer IoT. Industrial IoT examples include: manufacturing equipment and medical devices. Network Connectivity in the things (devices) is achieved through: wireless/wired, Wi-Fi, Bluetooth, ZigBee, VPN and Cellular 2G/3G/LTE/4G. Thread as an alternative for home automation applications and Whitespace TV technologies being implemented in major cities for wider area IoT-based use cases. Depending on the application, factors such as range, data requirements, security, power demands and battery life will dictate the choice of one or some form of combination of the technologies.

A major part of the IoT is not so much about smart devices, but about sensors and actuators. \textbf{Sensors} measure physical inputs and transform them into raw data; \textbf{actuators} act on the signal from the sensors and convert it into output, which is then digitally storable for access and analysis.  These tiny innovations can measure anything from temperature, force, flow and position, to light intensity and then can be attached to everything from smart phones to the medical devices and then record \& send data back into the cloud.  This will allow companies to collect more and more specific feedback on how products or equipment are used, when and how they break, and even \textbf{predict} what users might want in the future. Smart-phone would not have been smart if it does not have an array of sensors embedded in it. A typical smart-phone is equipped with five to nine sensors, depending on the model.


\section{Alliance with Big Data}

The true value of IoT is not in the internet connected devices themselves; the value lies in making context-aware and relevant data and turning the result into enterprise-grade, tangible, \textbf{actionable} business insights. The IoT and big data are clearly intimately connected: billions of internet-connected things will, by definition, generate massive amounts of data. As the Things turn more digital, IoT will analyze - variety sources (structured and unstructured), types of data (text, audio, video, image and binary) and respond intelligently in real time. The \textbf{cloud} brings needed agility, scalability, global reach and reliability to an IoT platform. Needed scalability can be achieved by using (1) Cloud Centric IoT - Good choice for low-cost things where data can easily be moved, with few ramifications (2) Edge Analytics - Ideal for things producing large volumes of data that are difficult, costly or sensitive to move (3) Distributed Mesh Computing - \textbf{Future-ready} multi-party things automatically collaborate with privacy intact.

Big data, meanwhile, is characterized by \textbf{four Vs} - volume, variety, velocity and veracity. That is, big data comes in large amounts (volume), with a mixture of structured, semi-structured and unstructured data (variety), arrives at (often real-time) speed (velocity) and can be of uncertain provenance (veracity). Such information is unsuitable for processing using traditional SQL-queried relational database management systems (RDBMSs), which is why a constellation of alternative tools -- notably Apache's open-source \textbf{Hadoop} distributed data processing system, various \textbf{NoSQL} databases and a range of business intelligence platforms - have evolved to serve such a complex process.


\section{Conclusion}

To create a connected IoT ecosystem that maximizes business value, we need to connect: technologies, data, process, insight, action and people. The \textbf{T} of IoT is clearly important, but too often, it is the only area of focus when examining IoT in business. The \textbf{Things} are only the means to an end as entities that can capture data measuring physical conditions or sometimes as actuators to affect the system. The rest of the systems need to be instrumented to leverage the data: communicating it to the right place for action - whether the cloud, data center, or edge - and then using analytics to understand data patterns and craft a response to fix or optimize. For the IoT to work effectively, all the challenges around regulatory, legal, privacy and cybersecurity must be addressed; there needs to be a framework within which things (devices) and applications can exchange data securely over wired or wireless networks. To address these challenges, one key player, \textbf{OneM2M} issued Release 1, a set of 10 specifications covering requirements, architecture, Application Programming Interfaces (API) specifications, security solutions and mapping to common industry protocols.

\begin{acks}

  The authors would like to thank 

\end{acks}

\bibliographystyle{ACM-Reference-Format}
\bibliography{report} 

\end{document}
