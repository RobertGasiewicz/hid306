\documentclass[sigconf]{acmart}

\usepackage{hyperref}

\usepackage{endfloat}
\renewcommand{\efloatseparator}{\mbox{}} % no new page between figures

\usepackage{booktabs} % For formal tables

\settopmatter{printacmref=false} % Removes citation information below abstract
\renewcommand\footnotetextcopyrightpermission[1]{} % removes footnote with conference information in first column
\pagestyle{plain} % removes running headers

\begin{document}
\title{Big Data and the Internet of Things (IoT)}


\author{Murali Cheruvu}
\orcid{1234-5678-9012}
\affiliation{%
  \institution{Institute for Clarity in Documentation}
  \streetaddress{P.O. Box 1212}
  \city{Dublin} 
  \state{Ohio} 
  \postcode{43017-6221}
}
\email{trovato@corporation.com}

% The default list of authors is too long for headers}
\renewcommand{\shortauthors}{B. Trovato et al.}


\begin{abstract}
This paper provides an introduction of Internet of Things (IoT) and how Big Data can effectively improve the IoT process.
\end{abstract}

\keywords{i523}


\maketitle

\section{Introduction}

The Internet of things (IoT) is the network of physical devices, vehicles, and other items embedded with electronics, software, sensors, actuators, and network connectivity which enable these objects to collect and exchange data.

\section{IoT Examples - Consumer and Industrial IoT}

Sensors will be one of the key drivers of IoT expansion. Sensors measure physical inputs and transform them into raw data, which is then digitally storable for access and analysis. Miniaturization has enabled sensors to integrate into smart devices, expanding their capacity beyond data measurement and analytics to transmitting information via the internet.6 In today’s context, sensors can measure anything from temperature, force, pressure, flow and position, to light intensity. And they are being embedded into everything, from
electricity networks, roads and other infrastructure, to mobile, wearable, home automation and security devices.
Among the most discussed applications for sensors are all things smart: cities, environment, water, metering, security and emergency services, retail, logistics, industrial control, agriculture, farming, domestic and home automation, and e-health.7 Within the M&E industry, companies are using categories of sensors such as inertial, motion and image sensors used in animation, gaming, video images, camera stabilization, sports and 3-D. At the same time, multichannel video programming distributors (MVPDs) are experimenting with other areas of the IoT ecosystem — cloud computing, transmission and spectrum enhancement.


\section{Need of Big Data}

Cloud Centric IoT – good choice for low-cost “things” where data can easily be moved, with few ramifications
Edge Analytics – Ideal for “things” producing large volumes of data that are difficult, costly or sensitive to move
Distributed Mesh Computing – Multi-party “things” automatically collaborate with privacy intact


\section{Building blocks of IOT Platform}

•	People. Your teams will have to think creatively about how they can achieve specific goals, and then refine to generate practical and feasible applications. People don’t always like change, but they’ll have to embrace IoT, because, whether they are conscious of it or not, IoT will increasingly feed into the applications they use daily. Human and organizational behavior is critical to realizing the value of new approaches, and it’s particularly important in shifting an organization to demonstrate clearly what will change, how it affects people, and what they stand to gain from IoT applications. 

•	Process. Beyond thinking differently, acting differently will become necessary. Enterprises should explore where more and faster insights can lead to new responses, and then figure out what can be automated. It’ll be crucial to standardize and institutionalize the ways that various business stakeholders carry out their activities. The new behaviors stemming from new IoT processes will have to be built into job definitions and operational guides to gain the IoT efficiencies. 

•	Things. The “T” of IoT is clearly important, but too often, it’s the only area of focus when examining IoT in business. The Things are only the means to an end as objects that can capture data measuring physical conditions or sometimes as actuators to affect the system. The rest of the systems need to be instrumented to leverage the data: communicating it to the right place for action—whether the cloud, data center, or edge—and then using analytics to understand what the data is saying and craft a response to fix or optimize.


\section{IOT Challenges}

Regulatory – 
Privacy – 
In almost every consumer survey, privacy emerges as the No.1 concern. In an IDC US consumer survey, close
to 55% of respondents identified “ensuring my privacy” as their top expectation of third-party providers of home
automation services.22 Similarly, a Forrester’s survey of global enterprise decision-makers identified security and privacy among the top five concerns to IoT adoption and growth.23 Privacy is a major challenge that organizations need to overcome as the IoT ecosystem seeks to collect enormous amounts of data and contextual inputs from sensors and other IoT solutions.

Legal – 

Intellectual property rights
One of the biggest questions on everyone’s mind is: Who owns the data? Is it the company that manufactures the sensor, the company that manufactures the device, the individual whose data is being measured and collected? Certainly, EU legislators are emphasizing the rights of the individual to own their data, but this is not the case in all jurisdictions. Even in instances where data ownership is clear, the duration for which owners can own the rights of collected data still needs to be addressed.

Scalability – 
Standards -  
For IoT to progress, connectivity standards need to evolve. Similar to the communication challenges that
people from different parts of the world face, in today’s IoT ecosystem, devices, sensors, machines and people
are often speaking completely different languages with one another. Without a common language or standard of implementation, IoT will remain limited in its application.

Cybersecurity – 


\section{Conclusion}

\begin{acks}

  The authors would like to thank 

\end{acks}

\bibliographystyle{ACM-Reference-Format}
\bibliography{report} 

\end{document}
