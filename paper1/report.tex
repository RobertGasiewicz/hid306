\documentclass[sigconf]{acmart}

\usepackage{hyperref}

\usepackage{endfloat}
\renewcommand{\efloatseparator}{\mbox{}} % no new page between figures

\usepackage{booktabs} % For formal tables

\settopmatter{printacmref=false} % Removes citation information below abstract
\renewcommand\footnotetextcopyrightpermission[1]{} % removes footnote with conference information in first column
\pagestyle{plain} % removes running headers

\begin{document}
\title{Internet of Things (IoT) alliance with Big Data}


\author{Murali Cheruvu}
\orcid{xxxx-xxxx-xxxx}
\affiliation{%
 \institution{Indiana University}
 \streetaddress{3209 E 10th St}
 \city{Bloomington} 
 \state{Indiana} 
 \postcode{47408}
}
\email{mcheruvu@iu.edu}

% The default list of authors is too long for headers}
\renewcommand{\shortauthors}{M. Cheruvu}


\begin{abstract}
This paper provides an introduction to Internet of Things (IoT) and how Big Data can effectively improve IoT process.
\end{abstract}

\keywords{i523, hid306, Internet of Things, IoT, Big Data, Sensors, Actuators, Analytics, Data Science}


\maketitle

\section{Introduction}

The Internet of things ({\em IoT}) is the network of physical devices, vehicles, and other items embedded with electronics, software, sensors, actuators, and network connectivity which enable these objects to collect and exchange data\cite{1_wiki_iot}. Devices of all types - cars, manufacturing equipment, medical devices and more - have become smarter, opening up the need for their connectivity with the internet. Today, over 50\% of IoT activity is centered in manufacturing, transportation, smart city, consumer applications like home automation and wearable gadgets, but within five years all industries will have rolled out IoT initiatives. {\em Gartner, Inc}. forecasts that 8.4 billion connected things will be in use worldwide in 2017 and will reach 20.4 billion by 2020\cite{2_Gartner}. 

\section{IoT Intuition}

The rise of IoT changes everything by enabling {\em smart} things. Products and environments are becoming smarter. Broadly speaking, two kinds of IoT are emerging: {\em Consumer IoT} and {\em Industrial IoT}. Products such as Apple Watch, Fitbit and Home Automation - TV, thermostats, alarm system, etc. are considered Consumer IoT. Industrial IoT are: manufacturing equipment and medical devices.

A few more examples of IoT include:

Track your activity levels - Using your smart-phone's range of sensors (accelerometer, gyro, video, proximity, compass, GPS, etc) and connectivity options (Cell, Wi-Fi, Bluetooth, etc.) you have a well equipped IoT device in your pocket that can automatically monitor your movements, location, and workouts throughout the day.

Get most out of your medication - The Proteus ingestible pill sensor is powered by contact with your stomach fluid and communicates a signal that determines the timing of when you took your medication and the identity of the pill. This information is transferred to a patch worn on the skin to be logged for you and your doctor's reference. Heart rate, body position and activity can also be detected.

Rolls-Royce is using Azure Stream Analytics and Power BI to link up sensor data from its engines with more contextual information like air traffic control, route data, weather and fuel usage to get a fuller picture of the health of its aircraft engines.

Smart Homes - A smart home is one in which devices have the capability to communicate with each other, as well as with their environment and the Internet. Smart homes enable owners to customize and control their home environments for increased security and efficient energy management. There are already hundreds of IoT technologies available to monitor and build smart homes.

Smart Cities - Smart surveillance, safer and automated transportation, smarter energy management systems and environmental monitoring are all examples of IoT applications for smart cities.

\section{Need of Big Data}

The true value of IoT is not in the internet connected devices themselves; the value lies in making context-aware and relevant data and turning the result into enterprise-grade, tangible, actionable business insights. The IoT and big data are clearly intimately connected: billions of internet-connected things will, by definition, generate massive amounts of data. As the Things turn more digital, IoT will analyze - variety sources (structured and unstructured), types of data (text, audio, video, image and binary) and respond intelligently in real time. 

Big data, meanwhile, is characterized by {\em four Vs} - volume, variety, velocity and veracity\cite{3_wiki_bigdata}. That is, big data comes in large amounts (volume), with a mixture of structured, semi-structured and unstructured data (variety), arrives at (often real-time) speed (velocity) and can be of uncertain provenance (veracity). Such information is unsuitable for processing using traditional SQL-queried relational database management systems (RDBMSs), which is why a constellation of alternative tools -- notably Apache's open-source {\em Hadoop} distributed data processing system, various {\em NoSQL} databases and a range of business intelligence platforms - have evolved to serve such a complex process.

Big Data is being generated at all times. Every digital process and social media exchange produces it. Systems, sensors and mobile devices transmit it. Much of this data is coming to us in an unstructured form, making it difficult to put into structured tables with rows and columns. To extract insights from this complex data, Big Data projects often rely on cutting edge analytics involving data science and machine learning. Computers running sophisticated algorithms can help enhance the veracity of information by sifting through the noise created by Big Data's massive volume, variety, and velocity.

\section{Alliance with Big Data}

To scale the needs of IoT, the strategy should include infrastructure and applications that process machine and sensor data, and leverage it accordingly. At the moment, IoT platforms are often custom-built functional architecture. Enterprises that take the first step into this new market should look for interoperability between existing systems and a new IoT operating environment.

The building blocks of the IoT platform must include:

{\em Things} -  A major part of the IoT is not so much about smart things (devices), but about sensors and actuators. {\em Sensors} measure physical inputs and transform them into raw data; {\em actuators} act on the signal from the sensors and convert it into output, which is then digitally storable for access and analysis.  These tiny innovations can measure anything from temperature, force, flow and position, to light intensity and then can be attached to everything from smart phones to the medical devices and then record \& send data back into the cloud. Smart-phone would not have been smart if it does not have an array of sensors embedded in it\cite{4_Wiley_Book}.  A typical smart-phone is equipped with five to nine sensors, depending on the model. 

Network Connectivity in the devices is achieved through: wireless/wired, Wi-Fi, Bluetooth, ZigBee, VPN and Cellular 2G/3G/LTE/4G. Thread as an alternative for home automation applications and Whitespace TV technologies being implemented in major cities for wider area IoT-based use cases. Depending on the application, factors such as range, data requirements, security, power demands and battery life will dictate the choice of one or some form of combination of the technologies. In March 2015, the Internet Architecture Board - a group within the Internet Society that oversees the technical evolution of the internet - released a guide to IoT networking. This outlined four common communication models used by IoT smart objects: Device-to-Device, Device-to-Cloud, Device-to-Gateway, and Back-End Data-Sharing\cite{5_Internet_Society}.

{\em Collaboration and Security} - Human and organizational behavior is critical in realizing the value of new approaches, and it is particularly important in shifting an organization to demonstrate clearly what will change, how it affects people, and what they stand to gain from IoT applications. Tons of collected IoT data could easily contain sensitive information about people and operations, and can even lose the control of critical systems. Beyond protecting personal privacy and business secrets, as more systems become automated, the risk of attacks becomes both more likely and more impactful. 

Devices themselves should be secured, as should operating systems, networks, and every other exposed piece of technology along the way. The roles of users, administrators, and managers should be individually defined with appropriate access and strong authentication embedded in the design. A multi-layered approach to security is essential, and it should have checks and balances to reinforce protection and, if necessary, diagnose any breaches. For the IoT to work effectively, all the challenges around regulatory, legal, privacy and cybersecurity must be addressed; there needs to be a framework within which things (devices) and applications can exchange data securely over wired or wireless networks. To address these challenges, one key player, {\em OneM2M} issued Release 1, a set of 10 specifications covering requirements, architecture, Application Programming Interfaces (API) specifications, security solutions and mapping to common industry protocols\cite{6_OneM2M}.

{\em Cloud} - The cloud brings needed agility, scalability, storage, processing, global reach and reliability to an IoT platform. Needed scalability can be achieved by using (1) Cloud Centric IoT - Good choice for low-cost things where data can easily be moved, with few ramifications (2) Edge Analytics - Ideal for things producing large volumes of data that are difficult, costly or sensitive to move (3) Distributed Mesh Computing - {\em Future-ready} multi-party things automatically collaborate with privacy intact. 

{\em Big Data Analytics} - The resulting flood of IoT sensor data must be understood and made actionable in the moment and over the long term. These data points will include structured data, unstructured data, and structured time series data, as well as a variety of analytical methods. Structured data might come from ERP systems and relational databases, such as supply chain and parts listings for automobile manufacturing. Exact specifications of each component are captured as transactional updates in tightly defined fields (part number, production lot, factory, etc.). Later the data may be extracted and joined with looser, unstructured text data like service records notes from car dealerships. And a time element may come in also as the service dates for oil changes and other periodic maintenance occur. Each data type introduces more information, but combined together will yield the secret of when a part failure is happening, help diagnose the origin of the problem, and suggest a preventative maintenance fix. 

Big data, in the context of the IoT, refers to analog inputs being converted to digital data and analyzed, and resulting in a response going back the other direction. Unlike some big data applications, the inputs should be at least semi-structured, but the sheer quantities and immediateness will raise other hurdles. Some analytics may need to be performed at the edge, some in the data center, and some in a cloud environment, depending on the trade-off of speed versus depth.

\section{Conclusion}

IoT is becoming disruptive yet inevitable for companies to welcome it. Creating a connected IoT ecosystem that maximizes business value, we need to collaborate: technologies, data, process, insight, action and people. The {\em T} of IoT is clearly important, but too often, it is the only area of focus when examining IoT in business. The Things are only the mean to an end as entities that can capture data measuring physical conditions or sometimes as actuators to affect the system. Rest of the systems need to be instrumented to leverage the data: communicating it to the right place for action - whether the cloud, data center, or edge - and then using analytics to understand data patterns and craft a response to fix or optimize. The goal of a connected IoT ecosystem is to get the most out of the Internet of Your Things in Your Context. Innovative organizations are starting to put this to use today.

\begin{acks}

  The author would like to thank 

\end{acks}

\bibliographystyle{ACM-Reference-Format}
\bibliography{report} 

\end{document}
