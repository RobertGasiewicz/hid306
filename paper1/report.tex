\documentclass[sigconf]{acmart}

\usepackage{graphicx}
\usepackage{hyperref}
\usepackage{todonotes}

\usepackage{endfloat}
\renewcommand{\efloatseparator}{\mbox{}} % no new page between figures

\usepackage{booktabs} % For formal tables

\settopmatter{printacmref=false} % Removes citation information below abstract
\renewcommand\footnotetextcopyrightpermission[1]{} % removes footnote with conference information in first column
\pagestyle{plain} % removes running headers

\newcommand{\TODO}[1]{\todo[inline]{#1}}

\begin{document}
	\title{The Internet of Things and Big Data Analytics}
	
	
	\author{Murali Cheruvu}
	\orcid{xxxx-xxxx-xxxx}
	\affiliation{%
		\institution{Indiana University}
		\streetaddress{3209 E 10th St}
		\city{Bloomington} 
		\state{Indiana} 
		\postcode{47408}
	}
	\email{mcheruvu@iu.edu}
	
	% The default list of authors is too long for headers}
	\renewcommand{\shortauthors}{M. Cheruvu}
	
	
	\begin{abstract}
		
		The Internet of Things, or IoT, is all about data from connected devices. Millions of consumer and industrial devices drive IoT growth and challenge with data volume and variety. Big Data Analytics helps combing through these high volumes of complex IoT data into meaningful business insights.
		
	\end{abstract}
	
	\keywords{i523, hid306, Internet of Things, IoT, Smart Devices, Sensors, Big Data Analytics}
	
	\maketitle
	

	
	\section{Introduction}	
	
	The Internet of Things ({\em IoT}) is the collection of devices having sensors, actuators, and Internet connectivity to gather, send and receive data. Devices of all types: cars, thermostats, implants for radio-frequency identification (RFID), pacemakers and more - have become smarter, opening up the need for their connectivity with the Internet. Most of the IoT activity is centered in transportation, smart environments, manufacturing, and consumer applications like wearable gadgets, but within three to five years all industries will roll-out some kinds of IoT initiatives. {\em Gartner, Inc}. predicts that 8.4 billion IoT devices will be in use by the end of 2017 and will reach 20.4 billion by 2020 \cite{gartner}. 
	
	\section{IoT Examples}
	
	The rise of IoT changes everything by enabling {\em smart} things. Products and environments are becoming smarter. Broadly speaking, two kinds of IoT are emerging: {\em Consumer IoT} and {\em Industrial IoT} \cite{iot}. Products such as Apple Watch, Fitbit, Smart TV, etc. are considered Consumer IoT. Examples of Industrial IoT are: manufacturing equipment and medical devices. A few more examples of IoT include:
	
	\subsection{Smartphones}
	Modern Smartphone is a good example of IoT device as it has sensors (GPS, compass, proximity, accelerometer, etc) and can connect to Intenet using Wi-Fi or Cell data. It can monitor location, workouts and movements throughout the day.
	
	\subsection{Smart Homes}
	
	Connected devices like security system, garage opener, thermostat and refrigerator need to communicate with each other to make the home as Smart Home. As an example, once the user reaches home, his car can communicate with the garage opener to open the garage and the main door can unlock automatically in response to the app installed on the Smartphone. The thermostat can adjust the temperature due to sensing his proximity. Refrigerator can reorder groceries using built-in scanning, sensors and actuators. 	
	
	\subsection{Smart Cities}
	
	Automated transportation with smart traffic lights, use of road sensors and smart parking, smarter energy consumption using smart grids and smart meters, environmental monitoring with usage of Wireless Sensor Network (WSN) are all examples of IoT applications for smart cities.
	
	\subsection{Smart Medical Alerts}
	
	Proteus invented the {\em smart pill} that has a sensor for tracking cardio-metabolic conditions of the patient. Once the patient takes this medication, the sensor starts sending signals to a patch worn on the skin, which logs patient diagnostic information and other metrics like activity patterns to the app of patience's Smartphone. Supervised doctor can also be notified about the patient with the details \cite{proteus}.
	
	\subsection{Smart Aircrafts}
	
	Rolls-Royce is building {\em Smart Aircrafts} to track engine performance, fuel usage, air traffic, routing details and weather conditions using sensors. The data from these sensors can be analyzed for improvements of the design of aircraft engines \cite{smart-aircraft}.	

	\section{Need of Big Data}
	
	The true value of IoT is not in just the Internet-connected devices; the value lies in making context-aware relevant data and converting the result to enterprise-grade, tangible and {\em actionable} business insights. The IoT and Big Data are highly interrelated: millions of Internet-connected devices will generate high volumes of data. As devices ({\em things}) turn more digital, IoT will analyze complex data-structures, and respond intelligently in real time. 
	
	Big Data is defined by {\em four Vs}: volume, variety, velocity and veracity \cite{big-data}. (a) Volume: Companies collect large amounts of data including transactions, sensor data and social media, and store them for later processing. (b) Variety: Data comes in various formats: structured, semi-structured and unstructured. Structured data usually come from RDBMS systems. Audio, video, binary and text documents are examples of semi-structured and unstructured data. Traditional relational databases (RDBMSs) will not be suitable for scale out distributed processing to handle such volume and variety. Alternatives like {\em Hadoop ecosystem}, with Distributed File System, Map, Reduce, etc. aspects, allows complex data processing. (c) Velocity: Data can come in batches, near-real time and real-time. Sensor data from medical devices might need immediate processing. (d) Veracity: Big Data Veracity refers to the noise and outlier data. Data mining will address these concerns using {\em data cleansing} and {\em normalization} techniques.
	

	\section{IoT Building Blocks}
	
	To scale the needs of IoT, the strategy should include infrastructure and applications that process and leverage machine and sensor data accordingly. At the moment, IoT platforms are often custom-built functional architecture. Enterprises that take the first step into this new market should look for interoperability between existing systems and a new IoT operating environment. The building blocks of an ideal IoT platform include:
	
	\subsection{Sensors and Actuators}
	A major part of the IoT is not so much about smart things (devices), but about sensors and actuators. Smartphone would not have been smarter if it does not have an array of sensors embedded in it. A typical smartphone is equipped with five to nine sensors, depending on the model. Both {\em Sensors} and  {\em actuators} are types of transducers which convert energy from one form to another, whereas sensors, are mainly applicable at the input and actuators take part in the output of a smart device \cite{wiley-book}. Sensors detect, quantify and convert recognized signal such as variations in pressure, heat or brightness into an analog or a digital electrical output that can easily be read and process. Thermometer senses and quantifies temperature into digital readable format, hence it is a sensor. Actuators are mechanical devices, such as switches, which produce signals by mechanical means. There are many types of sensors with endless capabilities to handle various use cases of IoT ranging from 
	simple consumer to advanced industrial scenarios.	
	
	
	\subsection{Network Connectivity}
	Wi-Fi and Cellular: 3G/LTE/4G are the most common network connectivity options for smart devices. Bluetooth and Zigbee are popular for short-range network communication. Thread technology is aimed at home automation applications and TV White Space, unused TV buffer channels, are providing broadband Internet access for wider area IoT-based use cases. Factors such as range, power usage, security and life of the battery will dictate the choice of which networking technologies to use. In March 2015, the Internet Architecture Board that oversees the technical evolution of the Internet released a guide to IoT networking. This outlined four standard communication models used by IoT smart devices: Device-to-Device, Device-to-Cloud, Device-to-Gateway, and Back-End Data-Sharing \cite{internet-society}.
	
		
	\subsection{Collaboration and Security}
	Human and organizational behavior is critical in realizing the value of IoT approaches, and it is particularly important in shifting an organization to demonstrate clearly what will change, how it affects people, and what they stand to gain from IoT applications. Tons of collected IoT data could easily contain sensitive information about people and operations, and can even lose control of critical systems. Beyond protecting personal privacy and business secrets, as more systems become automated, the risk of attacks becomes both more likely and more impactful. 
	
	Devices themselves should be secured, as should operating systems, networks and every other exposed piece of technology along the way. The roles of users, administrators and managers should be individually defined with appropriate access and strong authentication embedded in the design. A multi-layered approach to security is essential, and it should have checks and balances to reinforce protection and, if necessary, diagnose any breaches. For the IoT to work effectively, all the challenges around regulatory, legal, privacy and cybersecurity must be addressed; there needs to be a framework to exchange data securely over wired or wireless networks across devices. To address these challenges and for better IoT interoperability, one key player, {\em OneM2M} published Release 1, with 10 specifications covering requirements, architecture, security aspects, Application Programming Interface (API) specifications and mapping them to industry scenarios \cite{one-m2m}.
	
	\subsection{Cloud Computing}
	The cloud computing brings needed agility, scalability, storage, processing, global reach and reliability to an IoT platform. Flexible scalability can be achieved by using (a) Cloud Centric IoT: Good choice for low-cost things where data can easily be moved, with few ramifications (b) Edge Analytics: Ideal for things producing large volumes of data that are difficult, costly or sensitive to move, and (c) Distributed Mesh Computing: {\em Future-ready} multi-party devices automatically collaborate with privacy intact \cite{hp}. 
	
	\subsection{Big Data Analytics}
	Big Data Analytics, in the context of IoT, mainly refers to diagnostics, predictive maintenance, anomaly detection and reliability analysis using statistical tools and techniques with business acumen to explore hidden information from the sensor data. It applies data mining and machine learning algorithms to volume of data coming from multiple sources with various types of data formats. Typical data analytics workflow include: gathering structured and unstructured data, cleaning the data before modeling, evaluating and visualizing to make them usable for business decisions. Data modeling is, the heart of analytics, to better understand, quantify using statistical algorithms and then visualize the model to comply to the business context. Exploratory data analysis and predictive analytics are two major groups of tasks in the data modeling. Exploratory data analysis uses various techniques to provide useful textual and visual summaries of the characteristics of the data. Predictive analytics focuses on classification and numerical regression tasks. 
	
	
	\section{Conclusion}	
	
	
	The Connected Devices, also knows as the Internet of Things, are influencing people and corporations for ultimate functionality and superior usage of the Internet. Almost all the devices are or will be getting connected to the Internet. The goal of a connected IoT ecosystem is to get the most out of the Internet of your things in your context. Industrial IoT side, it is becoming disruptive yet inevitable for companies to welcome it. Creating a connected IoT ecosystem that maximizes business value, collaboration is need with technologies, data, process, insight and people. However, security and privacy will continue to be the key concerns to IoT growth. Innovative organizations are starting to address these concerns and pushing IoT devices to use today.
	
	\begin{acks}		
	
		The author would like to thank Dr. Gregor von Laszewski and the Teaching Assistants for their support and valuable suggestions.
		
	\end{acks}

	\bibliographystyle{ACM-Reference-Format}
	\bibliography{report} 
	

	
\end{document}
