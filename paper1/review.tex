\documentclass[sigconf]{acmart}

\usepackage{graphicx}
\usepackage{hyperref}
\usepackage{todonotes}

\usepackage{endfloat}
\renewcommand{\efloatseparator}{\mbox{}} % no new page between figures

\usepackage{booktabs} % For formal tables

\settopmatter{printacmref=false} % Removes citation information below abstract
\renewcommand\footnotetextcopyrightpermission[1]{} % removes footnote with conference information in first column
\pagestyle{plain} % removes running headers

\newcommand{\TODO}[1]{\todo[inline]{#1}}

\begin{document}
\title{The Internet of Things and Big Data}


\author{Murali Cheruvu}
\orcid{xxxx-xxxx-xxxx}
\affiliation{%
 \institution{Indiana University}
 \streetaddress{3209 E 10th St}
 \city{Bloomington} 
 \state{Indiana} 
 \postcode{47408}
}
\email{mcheruvu@iu.edu}

% The default list of authors is too long for headers}
\renewcommand{\shortauthors}{M. Cheruvu}


\begin{abstract}

The Internet of Things, or IoT, is all about data from connected devices. Millions of consumer and industrial devices drive IoT growth and challenge with data volume and variety. Big Data analytics helps combing through these high volumes of complex IoT data into meaningful business insights.

\end{abstract}

\keywords{i523, hid306, Data Science, Internet of Things, IoT, Smart Devices, Sensors, Actuators, Big Data Analytics, Cloud Computing}

\maketitle

\TODO{Paper has too many - in it}

\section{Introduction}

The Internet of Things ({\em IoT}) is the network of physical devices, vehicles, and other items embedded with electronics, software, sensors, actuators, and network connectivity which enable these objects to collect and exchange data\cite{1wikiiot}. Devices of all types - cars, thermostats, implants for radio-frequency identification (RFID), pacemakers and more - have become smarter, opening up the need for their connectivity with the internet. Today, over 50\% of IoT activity is centered in manufacturing, transportation, smart environments and consumer applications like wearable gadgets, but within five years all industries will have rolled out IoT initiatives. {\em Gartner, Inc}. forecasts that 8.4 billion connected things will be in use worldwide by end of 2017 and will reach 20.4 billion by 2020\cite{2Gartner}. 

\section{IoT Intuition}

The rise of IoT changes everything by enabling {\em smart} things. Products and environments are becoming smarter. Broadly speaking, two kinds of IoT are emerging: {\em Consumer IoT} and {\em Industrial IoT}. Products such as Apple Watch, Fitbit, Smart TV, etc. are considered Consumer IoT. Examples of Industrial IoT are: manufacturing equipment and medical devices. A few more examples of IoT include:

\subsection{Smartphones}
With smartphone's range of sensors (accelerometer, gyro, video, proximity, compass, GPS, etc.) and connectivity options (cell, wi-fi, bluetooth, etc.) user has well equipped IoT device that can automatically monitor movements, location and workouts throughout the day.

\subsection{Smart Homes}
An example of smart home enabled by IoT devices: The user arrives home and his car communicates with the garage to open the door. The thermostat is automatically adjusted to his preferred temperature due to sensing his proximity. He walks through his door as it unlocks in response to his smartphone or RFID implant. The home lighting is smartly turned on at dark.

\subsection{Smart Cities}
Smart surveillance, safer and automated transportation, smarter energy management systems and environmental monitoring are all examples of IoT applications for smart cities.

\subsection{Smart Medical Alerts}
The Proteus ingestible pill sensor is powered by contact with stomach fluid and communicates a signal that determines the timing of when patient took her medication and the identity of the pill. This information is transferred to a patch worn on the skin to be logged for the patient and her doctor's reference. Heart rate, body position and activity can also be detected accordingly.

\subsection{Smart Aircrafts}
Rolls-Royce is using Azure Cloud Stream Analytics and Power Business Intelligence (BI) to link up sensor data from its engines with more contextual information like air traffic control, route data, weather and fuel usage to get a complete report of the health of its aircraft engines.

\section{Need of Big Data}

The true value of IoT is not in the internet connected devices themselves; the value lies in making context-aware relevant data and converting the result into enterprise-grade, tangible and {\em actionable} business insights. The IoT and Big Data are intimately connected: billions of internet-connected things will, by definition, generate massive amounts of data. As the {\em things} turn more digital, IoT will analyze complex data structures and respond intelligently in real time. 

Big Data, meanwhile, is characterized by {\em four Vs} - volume, variety, velocity and veracity\cite{3wikibigdata}. That is, data come in large amounts ({\em volume}), with a combination of structured and unstructured data ({\em variety}), arrive at often real-time speed ({\em velocity}) and can be of uncertain source ({\em veracity}). Such information is unsuitable for processing using traditional SQL-queried relational database management systems (RDBMSs), which is why a cluster of alternative tools -- notably Apache's open-source {\em Hadoop} distributed data processing system, various {\em NoSQL} databases and a range of business intelligence ({\em BI}) platforms - have evolved to serve such a complex data process.


\section{IoT Building Blocks}

To scale the needs of IoT, the strategy should include infrastructure and applications that process and leverage machine and sensor data accordingly. At the moment, IoT platforms are often custom-built functional architecture. Enterprises that take the first step into this new market should look for interoperability between existing systems and a new IoT operating environment. The building blocks of an ideal IoT platform include:

\subsection{Sensors and actuators}
A major part of the IoT is not so much about smart things (devices), but about sensors and actuators. Smartphone would not have been smarter if it does not have an array of sensors embedded in it. A typical smartphone is equipped with five to nine sensors, depending on the model. {\em Sensors} measure physical inputs and transform them into raw data; {\em actuators} act on the signal from the sensors and convert it into output, which is then digitally storable for access and analysis.  These tiny innovations can measure anything ranging from temperature, force, flow, position to even light intensity then can be attached to everything from smartphones to medical devices and then record \& send data onto the cloud\cite{4WileyBook}. 

\subsection{Network Connectivity}
Network Connectivity in the devices is achieved through: wireless/wired, wi-fi, bluetooth, zigbee, VPN and cellular - 2G/3G/LTE/4G. Thread technology is emerging as an alternative for home automation applications and Whitespace TV technologies being implemented in major cities for wider area IoT-based use cases. Depending on the application, factors such as range, data requirements, security, power demands and battery life will dictate the choice of one or some form of combination of the technologies. In March 2015, the Internet Architecture Board - a group within the Internet Society that oversees the technical evolution of the internet - released a guide to IoT networking. This outlined four common communication models used by IoT smart devices: Device-to-Device, Device-to-Cloud, Device-to-Gateway, and Back-End Data-Sharing\cite{5InternetSociety}.

\subsection{Collaboration and Security}
Human and organizational behavior is critical in realizing the value of IoT approaches, and it is particularly important in shifting an organization to demonstrate clearly what will change, how it affects people, and what they stand to gain from IoT applications. Tons of collected IoT data could easily contain sensitive information about people and operations, and can even lose the control of critical systems. Beyond protecting personal privacy and business secrets, as more systems become automated, the risk of attacks becomes both more likely and more impactful. 

Devices themselves should be secured, as should operating systems, networks and every other exposed piece of technology along the way. The roles of users, administrators and managers should be individually defined with appropriate access and strong authentication embedded in the design. A multi-layered approach to security is essential, and it should have checks and balances to reinforce protection and, if necessary, diagnose any breaches. For the IoT to work effectively, all the challenges around regulatory, legal, privacy and cybersecurity must be addressed; there needs to be a framework within which devices and applications can exchange data securely over wired or wireless networks. To address these challenges and for better IoT interoperability, one key player, {\em OneM2M} issued Release 1, a set of 10 specifications covering requirements, architecture, Application Programming Interface (API) specifications, security solutions and mapping them to common industry protocols\cite{6OneM2M}.

\subsection{Cloud Computing}
The cloud computing brings needed agility, scalability, storage, processing, global reach and reliability to an IoT platform. Flexible scalability can be achieved by using (a) Cloud Centric IoT - Good choice for low-cost things where data can easily be moved, with few ramifications (b) Edge Analytics - Ideal for things producing large volumes of data that are difficult, costly or sensitive to move, and (c) Distributed Mesh Computing - {\em Future-ready} multi-party devices automatically collaborate with privacy intact. 

\subsection{Big Data Analytics}
Data Analytics involves statistical tools and techniques with business acumen to bring out hidden information from the data. Advanced types of data analytics include data mining, which involves sorting through large data sets to identify trends, patterns and relationships; predictive analytics, which seeks to predict customer behavior, equipment failures and other future events; and machine learning, an artificial intelligence technique that uses automated algorithms to churn through data sets more quickly than data scientists can do via conventional analytical modeling. Text mining provides a means of analyzing documents, emails and other text-based content. Big Data analytics applies data mining, predictive analytics and machine learning tools to volume of data coming from various sources with various types of data formats. 

Big Data analytics, in the context of the IoT, refers to sensor analog inputs being converted to digital data, analyzed, and resulting in a response going back to the device. Much of this data is in an unstructured form, making it difficult to put into structured tables with rows and columns. To extract valuable information from this complex data, Big Data applications often rely on cutting edge analytics involving data science. Distributed computers in the cloud running sophisticated algorithms can help enhance the veracity of information by data mining through the noise created by the massive volume, variety, and velocity. Some analytics may need to be performed using edge or mesh computing, some in the data center and some in a cloud environment, depending on the trade-off of speed versus depth. IoT analytics applications can help companies understand the IoT data at their disposal, with an eye toward reducing maintenance costs, avoiding equipment failures and improving business operations. 


\section{Conclusion}

Internet of Things shaping human life with greater connectivity and ultimate functionality, and all this is happening through ubiquitous networking to the Internet. There is seemingly no limit to what can be connected to the Internet. IoT will become more personal and predictive. The goal of a connected IoT ecosystem is to get the most out of the internet of your things in your context. Industrial IoT side, it is becoming disruptive yet inevitable for companies to welcome it. Creating a connected IoT ecosystem that maximizes business value, collaboration is need with technologies, data, process, insight, action and people. The {\em T} of IoT is clearly important, but too often, it is the only area of focus when examining IoT in business. Rest of the systems need to be instrumented to leverage the data: communicating it to the right place for action - whether the cloud, data center, or edge - and then using analytics to understand data patterns and craft a response to fix or optimize. However, security and privacy will be the top considerations for companies developing IoT devices. Innovative organizations are starting to put this to use today.


\bibliographystyle{ACM-Reference-Format}
\bibliography{report} 

\section{Bibtex Issues}
\DONE{Warning--no key, author in gartner}
\DONE{Warning--no author, editor, organization, or key in gartner}
\DONE{Warning--to sort, need author or key in gartner}
\DONE{Warning--no key, author in one-m2m}
\DONE{Warning--no author, editor, organization, or key in one-m2m}
\DONE{Warning--to sort, need author or key in one-m2m}
\DONE{Warning--no key, author in gartner}
\DONE{Warning--no key, author in gartner}
\DONE{Warning--no key, author in one-m2m}
\DONE{Warning--no key, author in one-m2m}
\DONE{Warning--no key, author in one-m2m}
\DONE{Warning--no author, editor, organization, or key in one-m2m}
\DONE{Warning--empty author in one-m2m}
\DONE{Warning--no key, author in gartner}
\DONE{Warning--no author, editor, organization, or key in gartner}
\DONE{Warning--empty author in Gartner}
\DONE{Warning--empty address in wiley-book}
\DONE{Warning--empty institution in internet-society}
\DONE{(There were 18 warnings)}
\section{Issues}

\TODO{no title}
\TODO{you do not use ``quotes'' properly}
\TODO{you need to {\em emphasize} and not ``quote''}
\TODO{this is cool}


\TODO{Have you written the report in the specified format?}
\TODO{Have you included an acknowledgement section?}
\TODO{Have you included the paper in the submission system (In our class it is git)?}
\TODO{Have you specified proper identification in the submission system. THis is typically a form or ASCII text that needs to be filled out (In our case it is a README.md file that includes a homework ID, names of the authors, and e-mails)?}
\TODO{Have you included all images in native and PDF format in the submission system?}
\TODO{Have you added the bibliography file that you managed (In our case jabref to make it simple for you)?}
\TODO{In case you used word have you also provided the jabref?}
\TODO{In case of a class and if you do a multi-author paper, have you added an appendix describing who did what in the paper?}
\TODO{Have you spellchecked the paper?}
\TODO{Are you usingaandtheproperly?}
\TODO{Have you made sure you do not plagiarize?}
\TODO{Is the title properly capitalized?}
\TODO{Have you not used phrases such as shown in the Figure below, but instead used as shown in Figure 3 when referring to the 3rd figure?}
\TODO{Have you capitalized “Figure 3”, “Table 1”, ... ?}
\TODO{Have you removed any figure that is not referred explicitly in the text (As shown in Figure ..)}
\TODO{Are the figure captions bellow the figures and not on top. (Do not include the titles of the figures in the figure itself but instead use the caption or that information?}
\TODO{When using tables have you put the table caption on top?}
\TODO{Make the figures large enough so we can read the details. If needed make the figure over two columns?}
\TODO{Do not worry about the figure placement if they are at a different location than you think. Figures are allowed to float. If you want you can place all figures at the end of the report?}
\TODO{Are all figures and tables at the end?}
\TODO{In case you copied a figure from another paper you need to ask for copyright permission. IN case of a class paper youmustinclude a reference to the original in the caption.}
\TODO{Do not use the word “I” instead use we even if you are the sole author?}
\TODO{Do not use the phrase “In this paper/report we show” instead use “We show”. It is not important if this is a paper or a report and does not need to be mentioned.}
\TODO{Do not artificially inflate your paper if you are bellow the page limit and have nothing to say anymore.}
\TODO{ If your paper limit is 12 pages but you want to hand in 120 pages, please check first ;-)}
\TODO{Donotusethecharacters \& \# \% \_ put a bakslash berfore them}
\TODO{If you want to say and do not use \& but use the word and.}
\TODO{Latex uses double single open quotes and double single closed quotes for quotes. Have you made sure you replaced them?}
\TODO{Pasting and copying from the Web often results in non ascii characters to be used in your text, please remove them and replace accordingly.}




\end{document}
